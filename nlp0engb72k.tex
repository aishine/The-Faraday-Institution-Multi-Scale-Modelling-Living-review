                         
Measurements of the open circuit voltage of Li-ion cells have been extensively used as a non-destruc
tive characterisation tool.                                                                         
Another technique based on entropy change measurements has also been applied for this purpose.      
More recently, both techniques have been used to make qualitative statements about aging in Li-ion c
ells.                                                                                               
One proposed cause of cell failure is point defect formation in the electrode materials.            
The steps in voltage profiles, and the peaks in entropy profiles are sensitive to order/disorder tra
nsitions arising from Li/vacancy configurations, which are affected by the host lattice structures. 
We compare the entropy change results, voltage profiles and incremental capacity (d$Q$/d$V$) obtaine
d from coin cells with spinel lithium manganese oxide (LMO) cathodes, \ce{Li_{1+y}Mn_{2-y}O_4},     
where excess Li $y$ was added in the range $0 \leq $y$ \leq 0.2$.                                   
A clear trend of entropy and d$Q$/d$V$ peak amplitude decrease with excess Li amount was determined.
The effect arises, in part, from the presence of pinned Li sites, which disturb the formation of the
 ordered phase.                                                                                     
We modelled the voltage, d$Q$/d$V$ and entropy results as a function of the interaction parameters a
nd the excess Li amount, using a mean field approach.                                               
For a given pinning population, we demonstrated that the asymmetries observed in the d$Q$/d$V$ peaks
 can be modelled by a single linear correction term.                                                
To replicate the observed peak separations, widths and magnitudes, we had to account for variation i
n the energy interaction parameters as a function of the excess Li amount, $y$.                     
All Li-Li repulsion parameters in the model increased in value as the defect fraction, $y$, increase
d.                                                                                                  
Our paper shows how far a computational mean field approximation can replicate experimentally observ
ed voltage, incremental capacity and entropy profiles in the presence of phase transitions.